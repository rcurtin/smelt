% In this section we must discuss the preparation of the ore for roasting,
% including the Pulverator 9000.

Before the chalcopyrite ore can be roasted, it has to be prepared.  In our
setting, which is a far cry from the continual 240-ton roasts of the Burra Burra
mine (taking on the order of 70 to 80 days for a completed roast), a high
priority was a fast roasting process (3 to 4 hours).  Because the roasting
process simply involves burning the chalcopyrite ore to release the sulfur
dioxide, increasing the exposed surface area of the ore can greatly decrease the
roasting time.  Therefore, smaller chunks of chalcopyrite ore were necessary.

The simplest method for achieving this result would be the cathartic use of a
sledgehammer against the large ore chunks, fragmenting them into pieces of the
desired size.  However, this imparts high kinetic energy into each fragment, and
high-velocity losses can be difficult to recover.  With a limited amount of ore,
loss prevention was a high priority, and consequently, this simple fracturing
technique was considered unfitting.

To this end, the development of small-scale techniques for ore crushing (or
`spalling') with minimal losses was necessary.
