\subsection{Theory}

% Detail what is actually happening in this process.

The roasting process (also called calcining) is a preparation step before the
actual smelting and copper recovery process.  With chalcopyrite ore
(\ce{CuFeS2}), the effect of the process is to release one of the two sulfur
atoms as sulfur dioxide.  The stochiometry of the reaction is given in Equation
\ref{eq:roasting}.

\begin{equation}
\cee{2 CuFeS2 + 3 O2 -> 2 FeO + 2 CuS + 2 SO2}
\label{eq:roasting}
\end{equation}

This process is necessary to guarantee good smelting results.  During the
smelting process, free sulfur will first combine with elemental copper; however,
when the elemental copper is exhausted, remaining sulfur will next combine with
iron, thereby reducing the purity of the produced copper matte.  Therefore, if
too much sulfur is present in the smelted ore, it affects the results of the
whole smelting process adversely \cite{peters1887}.

The temperatures necessary for this reaction to occur are approximately
400$^\circ$C to 600$^\circ$C.  It is important that the temperature of the roast
not exceed the melting point of copper (above 1000$^\circ$C), or the risk of
premature and inconsistent smelting arises.

In some industrial settings, techniques such as froth flotation are first
performed on the raw ore to improve its purity; however, this technology is not
easily reproduced by amateurs (due in part to its dependence on industrial
chemicals) and for the purpose of this investigation is ignored.

\subsection{Heap Roasting}

The most primitive form of roasting is referred to as `heap roasting', and
involves simply placing crushed ore on top of fuel (generally wood, but charcoal
can also work) and lighting the fuel.  As the reduction process begins, which is
made obvious by the odiferous release of sulfur dioxide, the sulfur in the ore
will become the fuel for the reaction, and will burn with a small, easily
recognizable electric blue flame.

This simple technique has been practiced since antiquity, but the process
leading to its development is not known.  One theory suggests that the
spontaneous combustion of sulfide ores (due to the natural decay of the
sulfides) led a curious primitive metallurgist to inspect the results and
discover deposits of what appeared to be copper, or more likely copper matte,
caused by the temperatures of the spontaneous roasting reaching the melting
point of copper \cite{peters1887}.  Another theory suggests that early workers
found the ore easier to break into pieces after it was burned
% \cite{somewhere!}
.  A third theory proposes that inobservant metallurgists confused copper ore
for gold ore and, in burning it and finding unexpected results, discovered that
copper could be produced instead through that method
% \cite{where the hell did this come from?  I think I have the theory wrong}
.

Many things must be considered when roasting copper ore.  First, the location:
the roasting process releases sulfur dioxide (and other gases, which will be
described shortly), which is particularly destructive to the local environment
and is an irritant when inhaled.  A large enough cloud of sulfur dioxide could
even cause asphyxiation (due to oxygen starvation).  The next thing to consider
is the size of the roast heap.  In an industrial setting, heaps containing in
excess of hundreds of tons of ore were used; the sulfur-releasing roast would
continue for potentially months.  When applied to smaller scales, the roast time
seems to decrease accordingly.  The third thing to be considered is the size of
ore chunks to be roasted in this fashion; this parameter can vary wildly, from
small chunks of one-and-a-half inches in diameter to larger chunks of a foot or
more in diameter.  The size of the chunk, of course, affects the roasting time
greatly.

Because the calcining process occurs only on those parts of the rocks exposed to
heat, the interior of ore chunks tends to remain unroasted.  Therefore, the heap
should be stirred frequently.  Some chunks may break open during this process,
thereby exposing their interiors to the necessary heat for the reaction to
occur.  Similar to the culinary task of meat preparation, ores tend to roast
somewhat quickly on the exterior, forming a crust of FeO (the familiar
reddish-brown rust), and then the progress of the roasting inwards can be slow.

The criteria for evaluating the completeness of a roast are generally
particularly dependent on the specific ore as well as the patience levels of the
metallurgist.  Realistically a heap of sulfide ore could be continually kept at
the correct temperature for calcination without ill effects, until sulfur
dioxide emissions had dropped to virtually zero, allowing the conclusion to be
drawn that the process is entirely complete; however, the resources necessary
for this perfection (in both time and labor) are generally infeasible.

\subsection{Byproducts of Roasting}

Because most copper ore is not pure (i.e. only chalcopyrite or chalcocite),
several other byproducts are produced in the roasting process.  A report on the
minor minerals present at the Burra Burra mine in Ducktown sheds some light on
the impurities generally found with chalcopyrite deposits \cite{slater1980}.  A
few of these impurities are listed below (countless more have been omitted).

\begin{itemize}
   \item Magnetite (\ce{Fe3O4})
   \vspace{-1em}
   \item Sphalerite (\ce{ZnS})
   \vspace{-1em}
   \item Actinolite (\ce{Ca2(Fe, Mg)5(Si8O22)(OH)2})
   \vspace{-1em}
   \item Calcite (\ce{CaCO3})
   \vspace{-1em}
   \item Biotite (\ce{K(Mg, Fe)3(AlSi3O10)(OH)2})
   \vspace{-1em}
   \item Pyrite (\ce{FeS2})
   \vspace{-1em}
   \item Arsenopyrite (\ce{FeAsS})
   \vspace{-1em}
   \item Galena (\ce{PbS})
\end{itemize}

Of particular interest in that list are galena (\ce{PbS}) and arsenopyrite
(\ce{FeAsS}), which contain two elements known to be harmful to humans -- lead
and arsenic.  The list of symptoms for either lead or arsenic poisoning are
generally long and unhappy; long-term effects of lead and arsenic exposure are
never beneficial, and some even suppose that lead exposure led indirectly to the
decline of the Roman Empire \cite{angier2007}. % Yeah, really.

The roasting of arsenopyrite results in the reaction shown in Equation
\ref{eq:arsenopyrite}.  The only non-harmful substance produced by that reaction
is \ce{Fe2O3}, a form of rust.  Arsenopyrite is generally present in copper ores
(though in trace quantities), so any metallurgist should be aware that during
the roasting process arsenic trioxide is released.  Care should be taken to avoid
inhalation of the sulfur dioxide plume; this undoubtedly contains trace amounts
of arsenic trioxide, which is certainly not beneficial from a medicinal
perspective.

\begin{equation}
\label{eq:arsenopyrite}
\cee{2FeAsS2 + 5O2 -> Fe2O3 + 2SO2 + As2O3}
\end{equation}

The other trace mineral of interest is galena, a lead sulfide.  The reaction
produced by the roasting of galena is specified in Equation \ref{eq:galena}.
The only gas produced by this reaction is sulfur dioxide; so, no gaseous lead
compounds are released.  Nonetheless, PbO is now present in the roasted ore.

\begin{equation}
\label{eq:galena}
\cee{2PbS + 3O2 -> 2PbO + 2SO2}
\end{equation}

Only two byproducts of roasting have been examined in detail here.  The exact
and full list of byproducts is dependent on the exact ore sample.  Even so, more
gases than just sulfur dioxide (and trace amounts of arsenic trioxide) are being
released; with high probability, other harmful gases are also being released.

\subsection{Environmental Impact}

The release of sulfur dioxide is well-known to be entirely detrimental to the
environment.  The combination of sulfur dioxide and moisture produces sulfuric
acid---or acid rain---which then falls from the sky and corrodes its
destination, whether that is plant life, geologic structures, or manmade
artifacts (or even lifeforms).

% Quick discussion of what happened at Ducktown and a reference to other
% sources.

\subsection{Experimental Setup}

For the purposes of this series of experiments, the ore was split into batches
of 15 to 40 pounds and heap roasted inside of a brick structure with forced
airflow, fueled by charcoal.  Different configurations of forced airflow and
fueling mechanisms were tried; however, unfortunately, the quality of results
could not be assessed until the smelting stage -- by which point the batches of
roasted ore had been recombined.  Four heap roast configurations are shown in
Figure \ref{fig:heaproast}.

% Subfigures will be necessary.
\begin{figure}[htb]
\caption{Four heap roast configurations, given in chronological order of use.}
\label{fig:heaproast}
\end{figure}

\subsubsection{Initial Proof-Of-Concept}

As a first venture into roasting, a few small rocks (up to 1.5 inches in
diameter) were piled into a paint roller tray along with a handful of charcoal
and lit with the assistance of charcoal lighter fluid.  No forced airflow was
present, other than that provided by the lungs of the experimenters.  In this
setting the blue flame of the calcination process was not observed; however, the
sudden increase in heat caused by the onset of calcination was clear, and the
resultant ignition of the nearby ground forced the cancellation of the
experiment.
