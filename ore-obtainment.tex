% First, we should discuss the important copper deposits on the planet.

% Then, we should focus more on a local copper deposit -- the Copper Basin in
% north Georgia and southeastern Tennessee.

% A quick history of the Burra Burra mine should have been given in the
% background; at this point, the museum should be mentioned, and its open mining
% area discussed.

\subsection{Important Copper Deposits}

Copper is not a particularly rare mineral; however, its deposits on the surface
are somewhat limited to certain regions.  Among these regions are \textbf{some
places!} and a region in the southeastern United States known as the `Copper
Basin', located near the intersection of Georgia, North Carolina, and Tennessee.

...

\subsection{The `Copper Basin'}

...

\subsection{Collection of Ore}

Using the facilities at the Burra Burra Mine Museum as well as tools purchased
at the local hardware store (a sledgehammer, an axe, a shovel, and several large
buckets for storage and transport), a large quantity of chalcopyrite-containing
ore was collected.

Large rocks were repeatedly struck until shattering; this process was repeated
until the largest remaining chunks of ore were approximately 10 to 12 inches
across.  Then, each chunk was inspected.  A chunk was considered to contain a
high percentage of metallic material if visual inspection revealed high
quantities of small, reflective metallic crystals.  However, this strategy for
high-grade ore selection may not have been optimal.  Unfortunately, due to
museum-imposed timing constraints, a more accurate selection process was
infeasible.  A comparison of a high-grade ore to a low-grade ore (according to
our primitive heuristic) is given in Figure \ref{fig:ore-comparison}.  Any
chunks smaller than 10 to 12 inches across considered `high-grade' by our simple
methodology was collected, down to small gravel-sized pieces.

\begin{figure}[htb]
% subfigures will need to be used here.

\caption{Comparison of high-grade and low-grade ores (according to a primitive
heuristic).}
\label{fig:ore-comparison}

\end{figure}

In total, six ten-gallon buckets were filled with what was believed to be high
quality chalcopyrite ore.  In addition, because the Burra Burra Mine Museum also
made flux available (mostly quartz), two duffel bags were filled with flux, as
well as slag samples from earlier Burra Burra mine operations.  The weight of
the collected chalcopyrite ore was determined to be $440.4$ pounds, and the
weight of the collected flux was $138.0$ pounds.  This ore was transported back
to an undisclosed location north of the Georgia Tech campus.
